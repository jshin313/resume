\documentclass{article}
\usepackage{titlesec}
\usepackage{titling}
\usepackage{geometry}
\usepackage{mathptmx}
\geometry{
 a4paper,
 total={170mm,267mm},
 left=20mm,
 right=20mm,
 top=10mm,
}
\usepackage{hyperref}
\hypersetup{
    colorlinks=true,
    linkcolor=black,
    filecolor=black,      
    urlcolor=black,
}
\pagenumbering{gobble} % Remove page numbering


\titleformat{\section}
{\large\bfseries}
{}
{0em}
{}[\titlerule]

\titleformat{\subsection}[runin]
{\normalsize\bfseries}
{}
{0em}
{}

\titlespacing{\section}
{0em}{1em}{1em}
\titlespacing{\subsection}
{0em}{0em}{1em}

\renewcommand{\maketitle}{
    \begin{center}
        {\huge\bfseries
        \theauthor}
    \end{center}
    \begin{center}
        \vspace{1em}
        \leavevmode
        \href{https://linkedin.com/in/jacob-shin}{linkedin.com/in/jacob-shin} $\bullet$ \href{https://github.com/jshin313}{github.com/jshin313} $\bullet$ \href{https://jacobshin.com}{jacobshin.com} $\bullet$ \href{jacobshin313@gmail.com}{jacobshin313@gmail.com} $\bullet$ 267 393 0368
    \end{center}
}

\begin{document}

\newcommand{\headerline}[3]{%
    \noindent
    \makebox[0pt][l]{#1}%
    \makebox[\textwidth][c]{#2}%
\makebox[0pt][r]{#3}}

\titlespacing{\headerline}
{0em}{0em}{1em}

\title{R\'esum\'e}
\author{Jacob Shin}

\maketitle
\section{Education}
\headerline{\textbf{Temple University}}{\textbf{BS in Computer Science (3.97 GPA)}}{Aug 2020 - May 2024}
\begin{itemize}
    \item President Scholar: Awarded a full tuition scholarship based on academic merit
    \item Temple Association for Computing Machinery (ACM), Temple Hack-a-Hardware / Computer Security Club 
\end{itemize}

\section{Skills}
\subsection{Programming Languages/Frameworks:}
C, C++, Python, Javascript, x86 ASM, Java
\subsection{Markup Languages:}
{\LaTeX}, Markdown, HTML, CSS
\subsection{Other:}
Linux, Bash, Git/Github, Tmux, Vim, Arduino, REST APIs, GDB (GNU Debugger), Binary Exploitation

\section{Experience}
\headerline{\textbf{Security Engineering Intern}}{\textbf{Security Innovation}}{June 2021 - Aug 2021}
\begin{itemize}
    \item Identified several vulnerabilities in client software by forcing software into states not intended by the developers (e.g. XSS, CSRF, Access Control Bypass, Session Fixation)
    \item Achieved arbitrary JavaScript execution, escalated privileges from a low privileged user to an administrator user, deleted other users' resources, and accessed data of other clients as a non-privileged user through the above vulnerabilities
    \item Wrote reports detailing the scope and severity of the vulnerabilities and recommended remediation steps

\end{itemize}
\headerline{\textbf{Undergraduate Research Assistant}}{\textbf{Temple University}}{January 2021 - May 2021}
\begin{itemize}
    \item Implemented a proxy to interface with the IFTTT (If This Then That) platform and IoT (Internet of Things) devices to detect anomalies that could indicate security concerns in a smart home
    \item Utilized Node.js and the Express Framework to implement a Service API based on the IFTTT specifications
\end{itemize}
\headerline{\textbf{Princeton Plasma Physics Lab Intern}}{Princeton, NJ}{Oct 2019 - Dec 2019}
\begin{itemize}
    \item Designed circuitry for a Langmuir probe, a device used to measure plasma properties like density and temperature
\end{itemize}

\section{Projects}

\textbf{\href{https://github.com/jshin313/CalcControlledBoat}{\underline{Calculator Controlled RC Boat}}} \hfill (C++, TI-BASIC, Arduino)
\begin{itemize}
    \item Utilized an Arduino and RF wireless modules to create the first ever calculator controlled, remotely controlled boat by interfacing a TI-84+ graphing calculator with a C++ library called ArTICL
    \item Enabled the library to support the TI-84+ calculator model by tracking down and fixing a bug in the implementation of the TI-Link protocol
\end{itemize}

\textbf{\href{https://github.com/jshin313/AquaQuant}{\underline{Water Utilization Dashboard}}} \hfill (React, C++, Flask, SQLite, Websockets, Material-UI, ESP8266 WiFi Module)
\begin{itemize}
    \item Developed a water usage tracking platform using vibration sensors to determine when water was being used and a wifi module to communicate with a Flask backend server via Websockets and a custom built REST API
    \item Awarded the best project "using IoT devices and technologies" prize by American Water at the Philly Codefest Hackathon out of 248 participants
\end{itemize}

\textbf{\href{https://github.com/jshin313/ti-authenticator}{\underline{TI-Authenticator: 2-Factor Authentication With a Calculator}}} \hfill (C, HMAC, SHA1, OTP)
\begin{itemize}
    \item Produced the first calculator app to provide rolling passcodes similar to Google Authenticator and Duo on a TI-84+ CE graphing calculator to enhance login security via 2-Factor Authentication
    \item Implemented the two types of One-Time Password (OTP) algorithms from scratch based on the \href{https://tools.ietf.org/html/rfc4226}{\underline{RFC 4226}} and \href{https://tools.ietf.org/html/rfc6238}{\underline{RFC 6238}} specifications based on a custom implementation of the HMAC algorithm (for learning purposes)
\end{itemize}

\textbf{\href{https://github.com/jshin313/unofficial-temple-covid-live-dashboard}{\underline{Web Scraper and Discord Bot}}} \hfill (Python, Flask, SQLite, Postgresql, Rust, Highcharts.js, Heroku)
\begin{itemize}
    \item Scraped the number of covid cases from the university website and displayed detailed cases vs. time graphs and bar charts with breakdowns of employees and on/off campus students via Flask and Highcharts
    \item Wrote a bot in Rust to interface to provide close to real time COVID data to various university Discord servers
\end{itemize}

\end{document}
