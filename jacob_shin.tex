\documentclass{article}
\usepackage{titlesec}
\usepackage{titling}
\usepackage{geometry}
\usepackage{mathptmx}
\geometry{
 a4paper,
 total={170mm,267mm},
 left=20mm,
 right=20mm,
 top=10mm,
}
\usepackage{hyperref}
\hypersetup{
    colorlinks=true,
    linkcolor=black,
    filecolor=black,      
    urlcolor=black,
}
\pagenumbering{gobble} % Remove page numbering


\titleformat{\section}
{\large\bfseries}
{}
{0em}
{}[\titlerule]

\titleformat{\subsection}[runin]
{\normalsize\bfseries}
{}
{0em}
{}

\titlespacing{\section}
{0em}{1em}{1em}
\titlespacing{\subsection}
{0em}{0em}{1em}

\renewcommand{\maketitle}{
    \begin{center}
        {\huge\bfseries
        \theauthor}
    \end{center}
    \begin{center}
        \vspace{1em}
        \leavevmode
        \href{https://linkedin.com/in/jacob-shin}{linkedin.com/in/jacob-shin} $\bullet$ \href{https://github.com/jshin313}{github.com/jshin313} $\bullet$ \href{https://jacobshin.com}{jacobshin.com} $\bullet$ \href{jacobshin313@gmail.com}{jacobshin313@gmail.com} $\bullet$ 267 393 0368
    \end{center}
}

\begin{document}

\newcommand{\headerline}[3]{%
    \noindent
    \makebox[0pt][l]{#1}%
    \makebox[\textwidth][c]{#2}%
\makebox[0pt][r]{#3}}

\titlespacing{\headerline}
{0em}{0em}{1em}

\title{R\'esum\'e}
\author{Jacob Shin}

\maketitle
\section{Education}
\headerline{\textbf{Temple University}}{Philadelphia, PA}{Aug 2020 - May 2024}
\begin{itemize}
    \item Bachelor of Science in Computer Science • GPA: 3.93
    \item Temple Association for Computing Machinery (ACM), Temple Hack-a-Hardware / Computer Security Club, Temple Data Science Initiative (TDSI)
\end{itemize}

\section{Skills}
\subsection{Programming Languages/Frameworks:}
C, C++, Python, Flask, Javascript, x86 ASM, Java, Selenium
\subsection{Markup Languages:}
{\LaTeX}, Markdown, HTML, CSS
\subsection{Other:}
Linux, Bash, Git/Github, Tmux, Vim (Neovim), Arduino, REST APIs, GDB (GNU Debugger), Binary Exploitation, Reverse Engineering

\section{Experience}
\headerline{\textbf{Princeton Plasma Physics Lab Intern}}{Princeton, NJ}{Oct 2019 - Dec 2019}
\begin{itemize}
    \item Designed circuitry for a Langmuir probe, a device used to measure plasma properties like density and temperature
\end{itemize}

\section{Projects}

\textbf{\href{https://github.com/jshin313/CalcControlledBoat}{\underline{Calculator Controlled RC Boat}}} \hfill (C++, TI-BASIC, Arduino)
\begin{itemize}
    \item Utilized an Arduino and RF wireless modules to create the first ever calculator controlled, remotely controlled boat by interfacing a TI-84+ graphing calculator with a C++ library called ArTICL
    \item Enabled the library to support the TI-84+ calculator model by tracking down and fixing a bug in the implementation of the TI-Link protocol
\end{itemize}

\textbf{\href{https://github.com/jshin313/AquaQuant}{\underline{Water Utilization Dashboard}}} \hfill (React, C++, Flask, SQLite, Websockets, Material-UI, ESP8266 WiFi Module)
\begin{itemize}
    \item Developed a water usage tracking platform using vibration sensors to determine when water was being used and a wifi module to communicate with a Flask backend server via Websockets and a custom built REST API.
    \item Awarded the best project "using IoT devices and technologies" prize by American Water at the Philly Codefest Hackathon out of 248 participants
\end{itemize}

\textbf{\href{https://github.com/jshin313/ti-authenticator}{\underline{TI-Authenticator: 2-Factor Authentication With a Calculator}}} \hfill (C, HMAC, SHA1, OTP)
\begin{itemize}
    \item Produced the first calculator app to provide rolling passcodes similar to Google Authenticator and Duo on a TI-84+ CE graphing calculator to enhanced login security via 2-Factor Authentication
    \item Implemented the two types of One-Time Password (OTP) algorithms from scratch based on the \href{https://tools.ietf.org/html/rfc4226}{\underline{RFC 4226}} and \href{https://tools.ietf.org/html/rfc6238}{\underline{RFC 6238}} specifications based on a custom implementation of the HMAC algorithm (for learning purposes)
\end{itemize}

\textbf{\href{https://github.com/jshin313/unofficial-temple-covid-live-dashboard}{\underline{Web Scraper and Discord Bot}}} \hfill (Python, Flask, SQLite, Postgresql, Rust, Highcharts.js, Heroku)
\begin{itemize}
    \item Scraped the number of covid cases from the university website and displayed detailed cases vs. time graphs and bar charts with breakdowns of employees and on/off campus students via Flask and Highcharts
    \item Wrote a bot in Rust to interface to provide close to real time COVID data to various university Discord servers
\end{itemize}

\section{Awards}
\subsection{Hackathon}
\begin{itemize}
    \item Drexel University's Philly Codefest Hackathon IoT Prize \hfill Dec 2020
\end{itemize}
\subsection{CTF (Capture the Flag Computer Security Competitions)}
\begin{itemize}
    \item 1st at castorsCTF20 (out of 500) $\bullet$ 2nd at OwlHacks RSM CTF $\bullet$ 4th at MetaCTF 2020 (out of 1017) $\bullet$ 4th at RACTF 2020 (out of 1047)
    \item 25th at PicoCTF 2019 (out of 11722) $\bullet$ 35th at TJCTF 2019 (out of 483) $\bullet$ 13th at MITRECTF 2019 (out of 262)
\end{itemize}
\end{document}
