\documentclass{article}
\usepackage{titlesec}
\usepackage{titling}
\usepackage{geometry}
\usepackage{mathptmx}
\geometry{
 a4paper,
 total={170mm,267mm},
 left=20mm,
 top=15mm,
}
\usepackage{hyperref}
\hypersetup{
    colorlinks=false,
    linkcolor=blue,
    filecolor=magenta,      
    urlcolor=blue,
}
\pagenumbering{gobble} % Remove page numbering

\titleformat{\section}
{\large\bfseries}
{}
{0em}
{}[\titlerule]

\titleformat{\subsection}[runin]
{\normalsize\bfseries}
{}
{0em}
{}

\titlespacing{\section}
{0em}{2em}{1em}
\titlespacing{\subsection}
{0em}{0em}{1em}

\renewcommand{\maketitle}{
    \begin{center}
        {\huge\bfseries
        \theauthor}
    \end{center}
    \begin{center}
        \vspace{1em}
        \leavevmode
        \href{https://linkedin.com/in/jacob-shin}{linkedin.com/in/jacob-shin} $\bullet$ \href{https://github.com/jshin313}{github.com/jshin313} $\bullet$ \href{https://jacobshin.com}{jacobshin.com} $\bullet$ \href{jacobshin313@gmail.com}{jacobshin313@gmail.com} $\bullet$ 267 393 0368
    \end{center}
}

\begin{document}

\title{R\'esum\'e}
\author{Jacob Shin}

\maketitle
\section{Education}
\subsection{Temple University} \hfill Expected: May 2024
\begin{itemize}
    \item BS in Computer Science • Honors Program • President's Scholar: Full-Tuition
    \item Courses: Introduction to Academic Computer Science, Math Concepts in Computing I (Discrete Mathematics)
\end{itemize}

\section{Experience}
\subsection{Princeton Plasma Physics Laboratory (PPPL) Intern}\hfill October 2019 - December 2019
\begin{itemize}
    \item Designed circuitry for a Langmuir probe, a device used to measure properties of plasmas like density and temperature.
\end{itemize}

\section{Projects}
\subsection{\href{https://github.com/jshin313/ti-authenticator}{\underline{TI-Authenticator: 2-Factor Authentication With a Calculator}}} \hfill (C, HMAC, SHA1, OTP)
\begin{itemize}
    \item Provided rolling passcodes similar to Google Authenticator and Duo except on a TI-84+ CE graphing calculator.
    \item Implemented One-Time Password (OTP) algorithms based on \href{https://tools.ietf.org/html/rfc4226}{\underline{RFC 4226}} (HOTP) and \href{https://tools.ietf.org/html/rfc6238}{\underline{RFC 6238}} (TOTP) specifications based on a custom implementation of the HMAC algorithm (for learning purposes).
\end{itemize}

\subsection{\href{https://github.com/jshin313/CalcControlledBoat}{\underline{Calculator Controlled RC Boat}}} \hfill (C++, TI-BASIC, Arduino)
\begin{itemize}
    \item Utilized an Arduino and RF wireless modules with a C++ library called ArTICL to interface with a TI-84+ graphing calculator to remotely control a boat.
    \item Tracked down and fixed a bug in the library that prevented the Arduino from properly interfacing with the specific calculator model.
\end{itemize}

\subsection{\href{https://github.com/jshin313/AquaQuant}{\underline{Water Utilization Dashboard}}} \hfill (React, Flask, Websockets, Material-UI, ESP8266 WiFi Module)
\begin{itemize}
    \item Developed a water usage tracking platform using vibration sensors to determine when water was being used and a wifi module to communicate with a Flask backend server via Websockets. 
\end{itemize}

\subsection{\href{https://github.com/jshin313/unofficial-temple-covid-live-dashboard}{\underline{COVID Data Web Scraper and Discord Bot}}} \hfill (Python, Flask, SQLite, Rust, Highcharts.js, Heroku)
\begin{itemize}
    \item Scraped the number of covid cases from the university website and displayed detailed cases vs. time graphs and bar charts with breakdowns of employees and on/off campus students via Flask and Highcharts.
    \item Wrote a bot in Rust to interface to provide close to real time COVID data to various university Discord servers.
\end{itemize}

\subsection{\href{https://ivyhub.org/decision-letters/}{\underline{College Rejection Simulator}}} \hfill (HTML, CSS, Javascript, Bootstrap, Netlify)
\begin{itemize}
    \item Helped over 30,000 high schoolers in 25 countries prepare for college rejection letters with an interactive simulation.
\end{itemize}




\section{Skills}
\subsection{Programming Languages/Frameworks:}
C, C++, Python, Flask, Javascript, x86 ASM
\subsection{Markup Languages:}
{\LaTeX}, Markdown, HTML, CSS
\subsection{Other:}
Linux, Bash, Git/Github, Tmux, Vim (Neovim), Arduino, REST APIs, GDB, Binary Exploitation, Basic Reverse Engineering

\section{Awards/Activities}
\subsection{CTF (Capture the Flag Computer Security Competitions):}
\begin{itemize}
    \item 1st at castorsCTF20 (out of 500) $\bullet$ 2nd at OwlHacks RSM CTF $\bullet$ 4th at MetaCTF 2020 (out of 1017) $\bullet$ 4th at RACTF 2020 (out of 1047)
    \item 25th at PicoCTF 2019 (out of 11722) $\bullet$ 35th at TJCTF 2019 (out of 483) $\bullet$ 13th at MITRECTF 2019 (out of 262)
\end{itemize} 
\subsection{Member: } Temple Association for Computing Machinery (ACM)
\subsection{Member: } Temple Hack-a-Hardware / Computer Security Club 
\end{document}
