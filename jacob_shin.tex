\documentclass[letterpaper,11pt]{article}
\usepackage{titlesec}
\usepackage{titling}
\usepackage{geometry}
\usepackage{mathptmx}
\geometry{
 left=20mm,
 right=20mm,
 top=12mm,
 bottom=12mm,
}
\usepackage{hyperref}
\hypersetup{
    colorlinks=true,
    linkcolor=black,
    filecolor=black,      
    urlcolor=black,
}
\pagenumbering{gobble} % Remove page numbering


\titleformat{\section}
{\large\bfseries}
{}
{0em}
{}[\titlerule]

\titleformat{\subsection}[runin]
{\normalsize\bfseries}
{}
{0em}
{}

\titlespacing{\section}
{0em}{1em}{1em}
\titlespacing{\subsection}
{0em}{0em}{1em}

\renewcommand{\maketitle}{
    \begin{center}
        {\huge\bfseries
        \theauthor}
    \end{center}
    \begin{center}
        \vspace{0em}
        \leavevmode
        \href{https://linkedin.com/in/jacob-shin}{linkedin.com/in/jacob-shin} $\bullet$ \href{https://github.com/jshin313}{github.com/jshin313} $\bullet$ \href{https://jacobshin.com}{jacobshin.com} $\bullet$ \href{jacobshin313@gmail.com}{jacobshin313@gmail.com} $\bullet$ 267 393 0368
    \end{center}
}

\begin{document}

\newcommand{\headerline}[3]{%
    \noindent
    \makebox[0pt][l]{#1}%
    \makebox[\textwidth][c]{#2}%
\makebox[0pt][r]{#3}}

\titlespacing{\headerline}
{0em}{0em}{1em}

\title{R\'esum\'e}
\author{Jacob Shin}

\maketitle
\section{Education}
\headerline{\textbf{Temple University}}{\textbf{BS in Computer Science (3.93 GPA)}}{Aug 2020 - May 2024}
\begin{itemize}
    \item Courses: Discrete Math I, Data Structures, Low Level Programming, Scientific Computing
\end{itemize}

\section{Skills}
\subsection{Programming Languages/Frameworks:}
C, C++, Java, Python, Javascript, x86 ASM, ez80 ASM
\subsection{Markup Languages:}
{\LaTeX}, Markdown, HTML, CSS
\subsection{Other:}
Linux, Bash, Git/Github, Tmux, Vim, Arduino, REST APIs, GDB (GNU Debugger), Binary Exploitation

\section{Experience}
\headerline{\textbf{Software Development Engineer Intern}}{\textbf{Amazon}}{May 2022 - Aug 2022}
\begin{itemize}
    \item Automated the retrieval and encryption of customer data, saving approximately 8 hours of engineering time per request
    \item Created a query API to filter through 10 terabytes of data using Typescript and Java
    \item Increased the concurrent user limit 20 fold for a secure Machine Learning Platform utilizing Access Control Lists (ACL), Virtual Private Cloud (VPC), Identity Access Management (IAM) policies, and encryption

\end{itemize}

\headerline{\textbf{Security Engineering Intern}}{\textbf{Security Innovation}}{June 2021 - Aug 2021}
\begin{itemize}
    \item Identified ~10 undiscovered vulnerabilities in 3 client projects by forcing software into states not intended by the developers (e.g. XSS, CSRF, Access Control Bypass, Session Fixation)
    \item Wrote and reviewed ~20 reports detailing the scope and severity of the vulnerabilities and recommended remediation steps 
\end{itemize}

\headerline{\textbf{Undergraduate Research Assistant}}{\textbf{Temple University}}{January 2021 - May 2021}
\begin{itemize}
    \item Implemented a proxy to interface with the IFTTT (If This Then That) platform and IoT (Internet of Things) devices to detect anomalies that could indicate security concerns in a smart home using Node.js
\end{itemize}
\headerline{\textbf{Princeton Plasma Physics Lab Intern}}{Princeton, NJ}{Oct 2019 - Dec 2019}
\begin{itemize}
    \item Created schematics for a Langmuir probe, which is used to measure plasma properties
\end{itemize}

\section{Projects}

\textbf{\href{https://github.com/jshin313/CalcControlledBoat}{\underline{Calculator Controlled RC Boat}}} \hfill (C++, TI-BASIC, Arduino)
\begin{itemize}
    \item Utilized an Arduino and RF wireless modules to create the first ever calculator controlled, remotely controlled boat by interfacing a TI-84+ graphing calculator with a C++ library called ArTICL
    \item Enabled the library to support the TI-84+ calculator model by tracking down and fixing a bug in the implementation of the TI-Link protocol
\end{itemize}

\textbf{\href{https://github.com/jshin313/ti-authenticator}{\underline{TI-Authenticator: 2-Factor Authentication With a Calculator}}} \hfill (C, HMAC, SHA1, OTP)
\begin{itemize}
    \item Produced the first calculator app to provide rolling passcodes similar to Google Authenticator and Duo on a TI-84+ CE graphing calculator to enhance login security via 2-Factor Authentication
    \item Implemented the two types of One-Time Password (OTP) algorithms from scratch based on the \href{https://tools.ietf.org/html/rfc4226}{\underline{RFC 4226}} and \href{https://tools.ietf.org/html/rfc6238}{\underline{RFC 6238}} specifications based on a custom implementation of the HMAC algorithm
\end{itemize}

\end{document}
