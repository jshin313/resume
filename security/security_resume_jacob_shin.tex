\documentclass{article}

\usepackage{titlesec}
\usepackage{titling}
\usepackage{geometry}
\usepackage{mathptmx}
\geometry{
 a4paper,
 total={170mm,267mm},
 left=20mm,
 top=10mm,
}
\usepackage{hyperref}
\hypersetup{
    colorlinks=true,
    linkcolor=black,
    filecolor=black,      
    urlcolor=black,
}
\pagenumbering{gobble} % Remove page numbering

\titleformat{\section}
{\large\bfseries}
{}
{0em}
{}[\titlerule]

\titleformat{\subsection}[runin]
{\normalsize\bfseries}
{}
{0em}
{}

\titlespacing{\section}
{0em}{2em}{1em}
\titlespacing{\subsection}
{0em}{0em}{1em}

\renewcommand{\maketitle}{
    \begin{center}
        {\huge\bfseries
        \theauthor}
    \end{center}
    \begin{center}
        \vspace{1em}
        \leavevmode
        \href{https://linkedin.com/in/jacob-shin}{linkedin.com/in/jacob-shin} $\bullet$ \href{https://github.com/jshin313}{github.com/jshin313} $\bullet$ \href{https://jacobshin.com}{jacobshin.com} $\bullet$ \href{jacobshin313@gmail.com}{jacobshin313@gmail.com} $\bullet$ 267 393 0368
    \end{center}
}

\newcommand{\headerline}[3]{%
    \noindent
    \makebox[0pt][l]{#1}%
    \makebox[\textwidth][c]{#2}%
\makebox[0pt][r]{#3}}

\titlespacing{\headerline}
{0em}{0em}{1em}

\begin{document}

\title{R\'esum\'e}
\author{Jacob Shin}

\maketitle
\section{Education}
\headerline{\textbf{Temple University}}{\textbf{BS in Computer Science (3.98 GPA)}}{Aug 2020 - May 2024}
\begin{itemize}
    \item Temple Association for Computing Machinery (ACM), Temple Hack-a-Hardware / Computer Security Club 
\end{itemize}

\section{Skills}
\subsection{Programming Languages:}
C, C++, Python, Javascript, x86 ASM
\subsection{Other:}
Linux, Git/Github, Tmux, (Neo)vim, Ghidra, GDB, Binary Exploitation, Basic Reverse Engineering, Pwntools

\section{Experience}
\headerline{\textbf{Security Engineering Intern}}{\textbf{Security Innovation}}{June 2021 - Aug 2021}
\begin{itemize}
    \item Identified several vulnerabilities in client software by forcing software into states not intended by the developers (e.g. XSS, CSRF, Access Control Bypass, Session Fixation)
    \item Achieved arbitrary JavaScript execution, escalated privileges from a low privileged user to an administrator user, deleted other users' resources, and accessed the data as a non-privileged user
    \item Wrote reports detailing the scope and severity of the vulnerabilities and recommended remediation steps 
    \item Conducted research exploring the security of platforms using the ez80 CPU and presented it to the company
\end{itemize}
\headerline{\textbf{Undergraduate Research Assistant}}{\textbf{Temple University}}{January 2021 - Present}
\begin{itemize}
    \item Implementing a proxy to interface with the IFTTT (If This Then That) platform and IoT (Internet of Things) devices.
    \item Utilized Node.js and the Express Framework to implement a Service API based on the IFTTT specifications
\end{itemize}
\headerline{\textbf{Intern}}{\textbf{Princeton Plasma Physics Laboratory}}{Oct 2019 - Dec 2019}
\begin{itemize}
    \item Created schematics for a Langmuir probe, which is used to measure plasma properties like density and temperature
\end{itemize}

\section{Projects}
\subsection{\textbf{\href{https://github.com/jshin313/CalcControlledBoat}{\underline{Calculator Controlled RC Boat}}}} \hfill (C++, TI-BASIC, Arduino)
\begin{itemize}
    \item Utilized an Arduino and RF wireless modules to create the first ever calculator controlled, remotely controlled boat by interfacing a TI-84+ graphing calculator with a C++ library called ArTICL
    \item Enabled the library to support the TI-84+ calculator model by tracking down and fixing a bug in the implementation of the TI-Link protocol
\end{itemize}

\subsection{\href{https://github.com/jshin313/ti-authenticator}{\underline{TI-Authenticator: 2-Factor Authentication With a Calculator}}} \hfill (C, HMAC, SHA1, OTP)
\begin{itemize}
    \item Produced the first calculator app to provide rolling passcodes similar to Google Authenticator and Duo on a TI-84+ CE graphing calculator to enhanced login security via 2-Factor Authentication
    \item Implemented the two types of One-Time Password (OTP) algorithms from scratch based on the \href{https://tools.ietf.org/html/rfc4226}{\underline{RFC 4226}} and \href{https://tools.ietf.org/html/rfc6238}{\underline{RFC 6238}} specifications based on a custom implementation of the HMAC algorithm (for learning purposes)
\end{itemize}

\subsection{\href{https://jacobshin.com/posts/}{ \underline{Personal Blog and Capture the Flag (CTF) Security Challenge Writeups}}}
\begin{itemize}
    \item Described the process of reversing using Ghidra (reverse engineering tool), bypassing exploitation mitigation techniques like NX (Non-executable stack) \& ASLR (Address space layout randomization), and leveraging Return Oriented Programming (ROP) to exploit a binary.
\end{itemize}

\section{Awards}
\subsection{CTF (Capture the Flag Computer Security Competitions):}
\begin{itemize}
    \item 1st at CalPoly $\bullet$ 1st at castorsCTF20 (out of 500)* $\bullet$ 2nd at OwlHacks RSM CTF $\bullet$ 4th at MetaCTF 2020 * $\bullet$ 4th at RACTF 2020 (out of 1047)* $\bullet$ 13th at MITRECTF 2019 (out of 262) $\bullet$ 25th at PicoCTF 2019 (out of 11722)*
    % \item  $\bullet$ 35th at TJCTF 2019 (out of 483)* $\bullet$ 13th at MITRECTF 2019 (out of 262)
\end{itemize} 
* denotes that I competed with a team

\end{document}
