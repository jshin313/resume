\documentclass{article}

\usepackage{titlesec}
\usepackage{titling}
\usepackage{geometry}
\usepackage{mathptmx}
\geometry{
 a4paper,
 total={170mm,267mm},
 left=20mm,
 top=10mm,
}
\usepackage{hyperref}
\hypersetup{
    colorlinks=true,
    linkcolor=black,
    filecolor=black,      
    urlcolor=black,
}
\pagenumbering{gobble} % Remove page numbering

\titleformat{\section}
{\large\bfseries}
{}
{0em}
{}[\titlerule]

\titleformat{\subsection}[runin]
{\normalsize\bfseries}
{}
{0em}
{}

\titlespacing{\section}
{0em}{2em}{1em}
\titlespacing{\subsection}
{0em}{0em}{1em}

\renewcommand{\maketitle}{
    \begin{center}
        {\huge\bfseries
        \theauthor}
    \end{center}
    \begin{center}
        \vspace{1em}
        \leavevmode
        \href{https://linkedin.com/in/jacob-shin}{linkedin.com/in/jacob-shin} $\bullet$ \href{https://github.com/jshin313}{github.com/jshin313} $\bullet$ \href{https://jacobshin.com}{jacobshin.com} $\bullet$ \href{jacobshin313@gmail.com}{jacobshin313@gmail.com} $\bullet$ 267 393 0368
    \end{center}
}

\newcommand{\headerline}[3]{%
    \noindent
    \makebox[0pt][l]{#1}%
    \makebox[\textwidth][c]{#2}%
\makebox[0pt][r]{#3}}

\titlespacing{\headerline}
{0em}{0em}{1em}

\begin{document}

\title{R\'esum\'e}
\author{Jacob Shin}

\maketitle
\section{Education}
\headerline{\textbf{Temple University}}{\textbf{BS in Computer Science (3.93 GPA)}}{Aug 2020 - May 2024}
\begin{itemize}
    \item President Scholar: Awarded a full tuition scholarship based on academic merit
    \item Temple Association for Computing Machinery (ACM), Temple Hack-a-Hardware / Computer Security Club, Temple Data Science Initiative (TDSI)
\end{itemize}

\section{Skills}
\subsection{Programming Languages:}
C, C++, Python, Javascript, x86 ASM
\subsection{Other:}
Linux, Git/Github, Tmux, (Neo)vim, Ghidra, GDB (GNU Debugger), Binary Exploitation, Basic Reverse Engineering, Wireshark, Pwntools, Nmap

\section{Experience}
\headerline{\textbf{Undergraduate Research Assistant}}{\textbf{Temple University}}{January 2021 - Present}
\begin{itemize}
    \item Implementing a proxy to interface with the IFTTT (If This Then That) platform and IoT (Internet of Things) devices.
    \item Utilized Node.js and the Express Framework to implement a Service API based on the IFTTT specifications
\end{itemize}
\headerline{\textbf{Intern}}{\textbf{Princeton Plasma Physics Laboratory}}{Oct 2019 - Dec 2019}
\begin{itemize}
    \item Designed circuitry for a Langmuir probe, a device used to measure plasma properties like density and temperature
\end{itemize}

\section{Projects}

\subsection{MITRE Embedded Security Challenge} \hfill (C, AES, AVR)
\begin{itemize}
    \item Designed a custom bootloader for an Atmega microcontroller with AES-CBC encryption and HMAC verification
    \item Attacked bootloaders from other teams by dumping flash via JTAG after finding out that fuse bits were incorrectly setup.
\end{itemize}

\subsection{\href{https://github.com/jshin313/ti-authenticator}{\underline{TI-Authenticator: 2-Factor Authentication With a Calculator}}} \hfill (C, HMAC, SHA1, OTP)
\begin{itemize}
    \item Produced the first calculator app to provide rolling passcodes similar to Google Authenticator and Duo on a TI-84+ CE graphing calculator to enhanced login security via 2-Factor Authentication
    \item Implemented the two types of One-Time Password (OTP) algorithms from scratch based on the \href{https://tools.ietf.org/html/rfc4226}{\underline{RFC 4226}} and \href{https://tools.ietf.org/html/rfc6238}{\underline{RFC 6238}} specifications based on a custom implementation of the HMAC algorithm (for learning purposes)
\end{itemize}

\subsection{\href{https://jacobshin.com/posts/}{ \underline{Personal Blog and Capture the Flag (CTF) Security Challenge Writeups}}}
\begin{itemize}
    \item Described the process of reversing using Ghidra (reverse engineering tool), bypassing exploitation mitigation techniques like NX (Non-executable stack) \& ASLR (Address space layout randomization), and leveraging Return Oriented Programming (ROP) to exploit a binary.
    \item Wrote a writeup on utilizing a tape-drive, emoji based assembly language to implement subtraction and xor from scratch with bitwise operators.
\end{itemize}

\subsection{Revere Engineering Malware Lab}
\begin{itemize}
    \item Learned reverse engineering techniques for reversing malware using Malware Unicorn's free, online reverse engineering workshops (Triage Analysis, Static Analysis, and Dynamic Analysis)
\end{itemize}

\section{Awards}
\subsection{CTF (Capture the Flag Computer Security Competitions):}
\begin{itemize}
    \item 1st at castorsCTF20 (out of 500) $\bullet$ 2nd at OwlHacks RSM CTF $\bullet$ 4th at MetaCTF 2020 (out of 1017) $\bullet$ 4th at RACTF 2020 (out of 1047)
    \item 25th at PicoCTF 2019 (out of 11722) $\bullet$ 35th at TJCTF 2019 (out of 483) $\bullet$ 13th at MITRECTF 2019 (out of 262)
\end{itemize} 

\end{document}
