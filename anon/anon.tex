\documentclass[letterpaper,11pt]{article}
\usepackage{titlesec}
\usepackage{titling}
\usepackage{geometry}
\usepackage{mathptmx}
\geometry{
 left=20mm,
 right=20mm,
 top=12mm,
 bottom=12mm,
}
\usepackage{hyperref}
\hypersetup{
    colorlinks=true,
    linkcolor=black,
    filecolor=black,      
    urlcolor=black,
}
\pagenumbering{gobble} % Remove page numbering


\titleformat{\section}
{\large\bfseries}
{}
{0em}
{}[\titlerule]

\titleformat{\subsection}[runin]
{\normalsize\bfseries}
{}
{0em}
{}

\titlespacing{\section}
{0em}{1em}{1em}
\titlespacing{\subsection}
{0em}{0em}{1em}

\renewcommand{\maketitle}{
    \begin{center}
        {\huge\bfseries
        \theauthor}
    \end{center}
    \begin{center}
        \vspace{0em}
        \leavevmode
        \href{https://youtu.be/dQw4w9WgXcQ}{linkedin.com/in/First-Last} $\bullet$ \href{https://youtu.be/dQw4w9WgXcQ}{github.com/jLast} $\bullet$ \href{https://youtu.be/dQw4w9WgXcQ}{FirstLast.com} $\bullet$ \href{FirstLast@gmail.com}{FirstLast@gmail.com} $\bullet$ 123 456 7890
    \end{center}
}

\begin{document}

\newcommand{\headerline}[3]{%
    \noindent
    \makebox[0pt][l]{#1}%
    \makebox[\textwidth][c]{#2}%
\makebox[0pt][r]{#3}}

\titlespacing{\headerline}
{0em}{0em}{1em}

\title{R\'esum\'e}
\author{First Last}

\maketitle
\section{Education}
\headerline{\textbf{Some University}}{\textbf{BS in Computer Science (3.97 GPA)}}{Aug 2020 - May 2024}
\begin{itemize}
    \item Courses: Discrete Math I, Data Structures, Computer Systems and Low Level Programming
    \item Some Association for Computing Machinery (ACM), Some Hack-a-Hardware / Computer Security Club 
\end{itemize}

\section{Skills}
\subsection{Programming Languages/Frameworks:}
C, C++, Python, Javascript, x86 ASM, Java
\subsection{Markup Languages:}
{\LaTeX}, Markdown, HTML, CSS
\subsection{Other:}
Linux, Bash, Git/Github, Tmux, Vim, Arduino, REST APIs, GDB (GNU Debugger), Binary Exploitation

\section{Experience}
\headerline{\textbf{Security Engineering Intern}}{\textbf{Small Security Company}}{June 2021 - Aug 2021}
\begin{itemize}
    \item Identified several vulnerabilities in client software by forcing software into states not intended by the developers (e.g. XSS, CSRF, Access Control Bypass, Session Fixation)
    \item Achieved arbitrary JavaScript execution, escalated privileges from a low privileged user to an administrator user, deleted other users' resources, and accessed the data as a non-privileged user through the above vulnerabilities
    \item Wrote reports detailing the scope and severity of the vulnerabilities and recommended remediation steps

\end{itemize}
\headerline{\textbf{Undergraduate Research Assistant}}{\textbf{Some University}}{January 2021 - May 2021}
\begin{itemize}
    \item Implemented a proxy to interface with the IFTTT (If This Then That) platform and IoT (Internet of Things) devices to detect anomalies that could indicate security concerns in a smart home using Node.js
\end{itemize}
\headerline{\textbf{DOE Lab Intern}}{Location, NJ}{Oct 2019 - Dec 2019}
\begin{itemize}
    \item Created schematics for a Langmuir probe, which is used to measure plasma properties like density
\end{itemize}

\section{Projects}

\textbf{\href{https://youtu.be/dQw4w9WgXcQ}{\underline{Calculator Controlled RC Boat}}} \hfill (C++, TI-BASIC, Arduino)
\begin{itemize}
    \item Utilized an Arduino and RF wireless modules to create the first ever calculator controlled, remotely controlled boat by interfacing a TI-84+ graphing calculator with a C++ library called ArTICL
    \item Enabled the library to support the TI-84+ calculator model by tracking down and fixing a bug in the implementation of the TI-Link protocol
\end{itemize}

\textbf{\href{https://youtu.be/dQw4w9WgXcQ}{\underline{Water Utilization Dashboard}}} \hfill (React, C++, Flask, SQLite, Websockets, Material-UI, ESP8266 WiFi Module)
\begin{itemize}
    \item Developed a water usage tracking platform using vibration sensors to determine when water was being used and a WiFi module to communicate with a Flask backend server via Websockets
    \item Awarded the best project "using IoT devices and technologies" prize by Company at the Location Codefest Hackathon out of 248 participants
\end{itemize}

\textbf{\href{https://youtu.be/dQw4w9WgXcQ}{\underline{TI-Authenticator: 2-Factor Authentication With a Calculator}}} \hfill (C, HMAC, SHA1, OTP)
\begin{itemize}
    \item Produced the first calculator app to provide rolling passcodes similar to Google Authenticator and Duo on a TI-84+ CE graphing calculator to enhance login security via 2-Factor Authentication
    \item Implemented the two types of One-Time Password (OTP) algorithms from scratch based on the \href{https://youtu.be/dQw4w9WgXcQ}{\underline{RFC 4226}} and \href{https://youtu.be/dQw4w9WgXcQ}{\underline{RFC 6238}} specifications based on a custom implementation of the HMAC algorithm
\end{itemize}

\end{document}
