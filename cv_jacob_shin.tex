\documentclass[letterpaper,11pt]{article}
\usepackage{titlesec}
\usepackage{titling}
\usepackage{geometry}
\usepackage{mathptmx}
\geometry{
 left=20mm,
 right=20mm,
 top=12mm,
 bottom=12mm,
}
\usepackage{hyperref}
\hypersetup{
    colorlinks=true,
    linkcolor=black,
    filecolor=black,      
    urlcolor=black,
}
\pagenumbering{gobble} % Remove page numbering


\titleformat{\section}
{\large\bfseries}
{}
{0em}
{}[\titlerule]

\usepackage{multicol}
\usepackage{etoolbox,refcount}

\newcounter{countitems}
\newcounter{nextitemizecount}
\newcommand{\setupcountitems}{%
  \stepcounter{nextitemizecount}%
  \setcounter{countitems}{0}%
  \preto\item{\stepcounter{countitems}}%
}
\makeatletter
\newcommand{\computecountitems}{%
  \edef\@currentlabel{\number\c@countitems}%
  \label{countitems@\number\numexpr\value{nextitemizecount}-1\relax}%
}
\newcommand{\nextitemizecount}{%
  \getrefnumber{countitems@\number\c@nextitemizecount}%
}
\newcommand{\previtemizecount}{%
  \getrefnumber{countitems@\number\numexpr\value{nextitemizecount}-1\relax}%
}
\makeatother    
\newenvironment{AutoMultiColItemize}{%
\ifnumcomp{\nextitemizecount}{>}{3}{\begin{multicols}{2}}{}%
\setupcountitems\begin{itemize}}%
{\end{itemize}%
\unskip\computecountitems\ifnumcomp{\previtemizecount}{>}{3}{\end{multicols}}{}}

\titleformat{\subsection}[runin]
{\normalsize\bfseries}
{}
{0em}
{}

\titlespacing{\section}
{0em}{1em}{1em}
\titlespacing{\subsection}
{0em}{0em}{1em}

\renewcommand{\maketitle}{
    \begin{center}
        {\huge\bfseries
        \theauthor}
    \end{center}
    \begin{center}
        \vspace{0em}
        \leavevmode
        \href{https://linkedin.com/in/jacob-shin}{https://linkedin.com/in/jacob-shin} $\bullet$ 
          \href{https://jacobshin.com}{jacobshin.com} $\bullet$ \href{mailto:jacobshin313@gmail.com}{jacobshin313@gmail.com} $\bullet$ +1 267 393 0368
    \end{center}
}

\begin{document}

\newcommand{\headerline}[3]{%
    \noindent
    \makebox[0pt][l]{#1}%
    \makebox[\textwidth][c]{#2}%
\makebox[0pt][r]{#3}}

\titlespacing{\headerline}
{0em}{0em}{1em}

\title{Jacob Shin CV}
\author{Jacob Shin}

\maketitle
\section{Education}
\headerline{\textbf{Temple University}}
{\textbf{Philadelphia, PA, United States}}{Aug 2020 - May 2024}
Bachelor of Science in Physics and Computer Science \\
Cumulative GPA: 3.93 out of 4.00


\section{Research Experience}
\headerline{\textbf{Undergraduate Research Assistant}}{\textbf{Temple University}}{August 2022 - Present}\\
\headerline{Department of Physics}{}{}
\headerline{Water Science Research Group}{}{}
\headerline{Advisor: Dr. Xifan Wu}{}

\begin{itemize}
    \item Currently analyzing density functional theory (DFT) simulation data to determine the role Van Der Waals forces play in water's unusal properties (e.g. negative thermal expansivity and density maximum at 4 $^\circ$C)
    \item Processed 5 Terabytes of raw simulation data on a high performance computing cluster by vectorizing computations and using parallel processes
\end{itemize}
\headerline{\textbf{Undergraduate Research Assistant}}{\textbf{Temple University}}{December 2021 - May 2022}
\headerline{Department of Physics}{}\\
\begin{itemize}
    \item Simulated the interactions of particles (e.g. electrons and protons) with detectors of different geometries and analyzed the resulting interactions using C++

\end{itemize}
\headerline{\textbf{Undergraduate Research Assistant}}{\textbf{Temple University}}{January 2021 - May 2021}
\headerline{Department of Computer Science}{}\\
\begin{itemize}
    \item Implemented a web program to interface with the IoT (Internet of Things) devices to detect anomalies that could indicate security concerns in a smart home
    \item Navigated a codebase with over 40k lines of code and added 10k lines of code
\end{itemize}

\section{Work Experience}
\headerline{\textbf{Amazon}}{Seattle, WA}{May 2022 - Aug 2022}
\headerline{Software Development Engineer Intern}{}{}
\begin{itemize}
    \item Created a Machine Learning (ML) Platform to automate the process of securely transferring ML data.
    % \item Automated the retrieval and encryption of customer data, saving approximately 8 hours of engineering time per request
    \item Created a query API to filter through 10 terabytes of data using Typescript and Java
\end{itemize}

\headerline{\textbf{Security Innovation, Inc.}}{Seattle, WA}{June 2021 - Aug 2021}
\headerline{Security Engineer Intern}{}{}
\begin{itemize}
    \item Wrote and reviewed 20 reports detailing the scope and severity of the vulnerabilities in code and recommended remediation steps 
    \item Conducted independent research exploring the security of platforms using the ez80 CPU and presented it to the company
\end{itemize}


\headerline{\textbf{Princeton Plasma Physics Lab}}{Princeton, NJ}{Oct 2019 - Dec 2019}
\headerline{Intern}{}{}
\begin{itemize}
    \item Created schematics for a Langmuir probe, which is used to measure plasma properties such as temperature and density based on the I-V (Current-Voltage) curve.
 \item Performed component selection for the Langmuir probe based on the specifications of the plasma parameters and the signal filtering requirements.
\end{itemize}
\pagebreak

\section{Skills}
\subsection{Programming Languages/Frameworks:}
Python, Jupyter Notebooks, Mathematica, Matlab, C, C++, Javascript
\subsection{Markup Languages:}
{\LaTeX}, Markdown, HTML, CSS
\subsection{Other:}
Linux, Bash, Git/Github, Vim, SSH/SCP
\subsection{Laboratory Skills:} Multisim, Soldering, Basic Machine Shop Training
\subsection{Human Languages:}
English

\section{Activities}
\begin{itemize}
    \item \textbf{Temple Robotics} - Contributed to the code base for the robot to be used in the NASA Robotics Mining/Lunabotics Competition and operated the mill in the machine shop to create components
    \item \textbf{Temple Data Science Club} - Created challenge problems for students to learn programming and computer security
    \item \textbf{Temple Physics Club} - Member
    \item \textbf{Schuylkill Center Wildlife Clinic} - Volunteer (2021-2022)
\end{itemize}


\section{Awards and Honors}
\begin{itemize}
    \item \textbf{Temple Presidential Scholarship} - Full Tuition Merit Scholarship for 4 years
    \item \textbf{Science Scholars Program} - Selective research program that offers a \$4,000 stipend per summer for research
    \item \textbf{Temple Dean's List} - Granted to the top 16\% of students: Fall 2020, Spring 2020, Fall 2021
    \item \textbf{Philly Codefest American Water IoT Prize} - Won ~\$1000 in prizes for the best IoT electronics and coding project
    \item \textbf{Temple Honors Program} - Selective program for high-achieving students which offers advising and advanced honors level classes
\end{itemize}

\section{Courses}
\begin{AutoMultiColItemize}
  \item Analytical Mechanics
  \item Optics
  \item Principles of Electric Circuits
  \item Introduction to Modern Physics
  \item Mathematical Physics
  \item Thermal Physics
  \item Electricity and Magnetism
  \item Scientific Computing III (Intro to Machine Learning)
  \item Classical Mechanics
  \item Physics 1 \& Physics 2
  \item Real \& Complex Analysis I
  \item Basic Concepts (Intro to Proofs)
  \item Differential Equations with Linear Algebra
  \item Calculus III (Multivariable and Vector Calculus)
  \item Computer Systems and Low Level Programming
  \item Data Structures
  \item Mathematical Concepts in Computing I (Discrete Mathematics)
  \end{AutoMultiColItemize}

\end{document}
