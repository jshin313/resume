\documentclass{article}

\usepackage{titlesec}
\usepackage{titling}
\usepackage{geometry}
\geometry{
 a4paper,
 total={170mm,267mm},
 left=20mm,
 top=15mm,
}
\usepackage{hyperref}
\hypersetup{
    colorlinks=true,
    linkcolor=blue,
    filecolor=magenta,      
    urlcolor=blue,
}
\pagenumbering{gobble} % Remove page numbering

\titleformat{\section}
{\large\bfseries}
{}
{0em}
{}[\titlerule]

\titleformat{\subsection}[runin]
{\normalsize\bfseries}
{}
{0em}
{}

\titlespacing{\section}
{0em}{2em}{1em}
\titlespacing{\subsection}
{0em}{0em}{1em}

\renewcommand{\maketitle}{
    \begin{center}
        {\huge\bfseries
        \theauthor}
    \end{center}
    \begin{center}
        \vspace{1em}
        \leavevmode
        https://github.com/jshin313 $\bullet$ https://www.jacobshin.com $\bullet$ jacobshin313@gmail.com $\bullet$ (267) 393 0368
    \end{center}
}

\begin{document}

\title{R\'esum\'e}
\author{Jacob Shin}

\maketitle
\section{Education}
\subsection{Temple University (College of Science and Technology)}\hspace{4em} Expected to Graduate May 2024
\begin{itemize}
    \item Bachelor of Science, Computer Science • Honors Program
    \item President's Scholar: Covers Full-Tuition (\$20,000/yr) • Temple Science Scholar
    \item Courses: Introduction to Academic Computer Science, Mathematical Concepts in Computing I Honors 
\end{itemize}

\section{Experience}
\subsection{Princeton Plasma Physics Laboratory (PPPL) Intern} (October 2019 - December 2019)
\begin{itemize}
    \item Learned to design an electronic circuit for a device called a Langmuir probe, an instrument used to measure properties like density and temperature of plasmas
\end{itemize}

\section{Projects}
\subsection{\href{https://github.com/jshin313/ti-authenticator}{TI-Authenticator: 2FA With a Calculator}} (C, HMAC, SHA1, OTP)
\begin{itemize}
    \item Provides rolling passcodes similar to Google Authenticator except on a graphing calculator
    \item Implements One-Time Password (OTP) algorithms for the TI-84+ CE graphing calculator based on \href{https://tools.ietf.org/html/rfc4226}{RFC 4226} (HOTP) and \href{https://tools.ietf.org/html/rfc6238}{RFC 6238} (TOTP)
\end{itemize}

\subsection{MITRE Embedded Security Challenge 2017} (C, AES, AVR)
\begin{itemize}
    \item Designed a bootloader for "Secure Firmware Distribution for Automotive Control" using an Atmega1284p microcontroller using HMAC verfication and AES-CBC encryption.
    \item Attacked other bootloaders from other teams by dumping flash via JTAG (after finding out that fuse bits were incorrectly setup)
    \item Learned about brownout attacks and side channel attacks
\end{itemize}

\subsection{\href{https://jacobshin.com/posts/castorsctf-babybof1pt2/}{Ret2LibC Buffer Overflow CTF Challenge Writeup}}
\begin{itemize}
    \item Wrote a writeup of how I solved the BOF (Buffer Overflow) CTF challenge for the RACTF challenge
    \item Described the process of reversing using Ghidra (reverse engineering tool), bypassing exploitation mitigation techniques, and leveraging Return Oriented Programming (ROP) to exploit a binary.
\end{itemize}

\subsection{Revere Engineering Malware}
\begin{itemize}
    \item Learned reverse engineering techniques for reversing malware using Malware Unicorn's free, online reverse engineering workshops (Triage Analysis, Static Analysis, and Dynamic Analysis)
\end{itemize}

\section{Skills}
\subsection{Programming Languages:}
C, C++, Python, Javascript, x86 ASM
\subsection{Markup Languages:}
{\LaTeX}, Markdown, HTML, CSS
\subsection{Other:}
Linux, Bash, Git/Github, Tmux, (Neo)vim, REST APIs, Ghidra, GDB, Binary Exploitation, Reverse Engineering, Pwntools

\section{Awards/Activities}
\subsection{4th Place: } RACTF 2020 Computer Security Competition
\subsection{1st Place: } castorsCTF20 Computer Security Competition
\subsection{25th HS Team: } PicoCTF 2019 Computer Security Competition
\subsection{35th Place: } TJCTF 2019 Computer Security Competition
\subsection{13th Place: } MITRE 2019 Cyber Challenge CTF
\subsection{\href{https://ctftime.org/user/66839}{Member: }} Pwn Intended CTF Team (Top 100 Globally)
\subsection{Member: } Temple Association for Computing Machinery (ACM)

\end{document}
