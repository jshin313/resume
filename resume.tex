\documentclass{article}

\usepackage{titlesec}
\usepackage{titling}
\usepackage{geometry}
\geometry{
 a4paper,
 total={170mm,267mm},
 left=20mm,
 top=8mm,
}
\usepackage{hyperref}
% \hypersetup{
%     colorlinks=true,
%     linkcolor=blue,
%     filecolor=magenta,      
%     urlcolor=cyan,
% }
\pagenumbering{gobble} % Remove page numbering

\titleformat{\section}
{\large\bfseries}
{}
{0em}
{}[\titlerule]

\titleformat{\subsection}[runin]
{\normalsize\bfseries}
{}
{0em}
{}

\titlespacing{\section}
{0em}{2em}{1em}
\titlespacing{\subsection}
{0em}{0em}{1em}

\renewcommand{\maketitle}{
    \begin{center}
        {\huge\bfseries
        \theauthor}
    \end{center}
    \begin{center}
        \vspace{1em}
        \leavevmode
        https://github.com/jshin313 $\bullet$ https://www.jacobshin.com $\bullet$ jacobshin313@gmail.com $\bullet$ (267) 393 0368
    \end{center}
}

\begin{document}

\title{R\'esum\'e}
\author{Jacob Shin}

\maketitle
\section{Education}
\subsection{Temple University}\hspace{31em} 2020 - 2024
\begin{itemize}
    \item Bachelor of Science, Computer Science • College of Science and Technology (CST) • Honors Program
    \item President's Scholar: Covers Full-Tuition (\$20,000/yr) • CST Science Scholar
    \item Courses: Introduction to Academic Computer Science, Honors Calculus I, Mathematical Concepts in Computing I Honors 
\end{itemize}

\subsection{Pennsbury High School}(4.65 Weighted GPA)\hspace{19em} 2016 - 2020
\begin{itemize}
    \item Academic Excellence in Computer Science -- Xerox Award for Innovation and Information Technology
\end{itemize}

\section{Experience}
\subsection{Princeton Plasma Physics Laboratory (PPPL) Intern} (Fall 2019 - Winter 2019)
\begin{itemize}
    \item Learned to design an electronic circuit for a device called a Langmuir probe, an instrument used to measure properties like density and temperature of plasmas
\end{itemize}

\section{Projects}
\subsection{\href{https://github.com/jshin313/ti-authenticator}{TI-Authenticator: 2FA With a Calculator}} (C, HMAC, SHA1, OTP)
\begin{itemize}
    \item Provides rolling passcodes similar to Google Authenticator except on a graphing calculator
    \item Implements One-Time Password (OTP) algorithms for the TI-84+ CE graphing calculator based on \href{https://tools.ietf.org/html/rfc4226}{RFC 4226} (HOTP) and \href{https://tools.ietf.org/html/rfc6238}{RFC 6238} (TOTP)
\end{itemize}

\subsection{\href{https://github.com/jshin313/CalcControlledBoat}{Calculator Controlled RC Boat}} (C++, TI-BASIC, Arduino)
\begin{itemize}
    \item Allows a graphing calculator to wirelessly control a boat
    \item Utilizes an Arduino and RF wireless modules with a C++ library called ArTICL to interface with a TI-84+ graphing calculator
\end{itemize}

\subsection{College Rejection Simulator} (HTML, CSS, Javascript, Bootstrap, Netlify)
\begin{itemize}
    \item Created a college rejection simulator with fake decision letters and college login portals to help high school seniors mentally prepare for their rejection
    \item Received 20,000 views within the first few days of the release
    \item Utilized my previous work from https://github.com/jshin313/preparetoberejected
\end{itemize}

\subsection{Wireless LED Marching Band Lights} (C++, Arduino, Group Project)
\begin{itemize}
    \item Created multi-coloured, music-synchronized lights for the drums and other instruments
    \item Redesigned an existing prototype by using more cost effective circuity in order to make the project feasible and fundable by the school
\end{itemize}

\section{Skills}
\subsection{Programming Languages:}
C, C++, Python, Javascript, x86 ASM
\subsection{Markup Languages:}
{\LaTeX}, Markdown, HTML, CSS
\subsection{Other:}
Linux, Bash, Git/Github, Tmux, (Neo)vim, Arduino

\section{Awards/Activities}
\subsection{1st Place: } castorsCTF20 Computer Security Competition
\subsection{4th Place: } RACTF 2020 Computer Security Competition
\subsection{Member: } Temple Association for Computing Machinery (ACM)

\end{document}
